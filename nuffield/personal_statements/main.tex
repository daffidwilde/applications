\documentclass{article}

\usepackage[margin=1.5cm, includefoot, footskip=30pt]{geometry}
\usepackage{titlesec}

\titleformat{\section}
  {\normalfont\Large\bfseries}{\thesection}{1em}{}[{\titlerule[0.8pt]}]

\title{Nuffield personal statements 2018}
\author{Henry Wilde}

\begin{document}
\maketitle

\section*{Placement description}
When we wish to compare the performance of two similar algorithms against one
another, it is difficult to discern why one may outperform the other with the
same data. In this project we will use Python to mimic the natural selection
process; constructing parts of a genetic algorithm that takes two algorithms and
generates populations of artificial datasets for which one algorithm outperforms
the other. The hope is then to discover what makes these datasets better-suited
for our preferred algorithm through data visualisation and analysis, while also
taking some time to analyse the genetic algorithm itself.


\section*{Student 1}

\subsection*{Summary}
\begin{itemize}
    \item Interested in astrophysics
    \item Seemingly no knowledge of Python or programming
\end{itemize}

\subsection*{Personal statement}

My passion for Astrophysics extends past the normal school curriculum and makes
me ideally suited for a placement with the Nuffield foundation. I recently
attended a show by Professor Brian Cox on cosmology, focused on the life cycle
of the universe and structure of galaxies at different points in the life span
of the universe. I was particularly interested in how the expansion of the
universe is being affected by the mass within it, and how the universe would
need to contain 10x the mass it does now in order for gravity to be able to
overcome the power of the expansion caused by dark energy, leading to an event
known as the “Big Crunch” and the possibility of a universe that goes through
cycles of expanding and collapsing over and over again. This contrasts with our
universe which seems to be dominated by the force of dark energy and, by looking
at the differences in redshift of distant galaxies, the universe seems to be
accelerating the rate it is expanding at. I found this talk fascinating and it
has inspired me to want to more, leading me to extend my own understanding of
these concepts and ideas.\\

Allied to this I have recently completed an astronomy course on Brilliant.org to
help bolster my knowledge of astronomy and gain a better understanding of the
tools and techniques used by astronomers. During this course, I learnt about the
use of the thermal emission spectrum of stars, Wien’s displacement and Stefan’s
law. As a consequence of this course I was further able to develop my critical
thinking and analytical ability to help me work through problems and come up
with results. My interest in astrophysics has further developed through the
“Seren Hub”, the Welsh governments programme for the most able sixth formers in
Wales. I recently attended a lecture at a Hub event by Dr Paul Roche about how
asteroids have affected the solar system and the threat they could pose to life
on Earth. This explored the idea that an asteroid impact may be the reason for
mercury’s reduced mantle sizes compared to its core, with 70\% of its mass being
in the iron core, compared with only 32\% of the mass of Earth’s core. Some
scientists believe it is the remains of a collision with a large planetary body
in the early formation of the solar system. This demonstrated to me the chaotic
nature of how the solar system formed and how the constant bombardment of the
proto-planets in the solar system led to irregularities such as Mercury’s
Mantle. This lecture helped me to apply logical thinking to understand why an
object is the way it is, and what possible events in its history could have
caused this to happen depending on the factors that it may have formed under or
what events it may have gone through during its life time.\\

I would be a good fit for a Nuffield research placement due to my utter
dedication to, and love of, astrophysics, along with my tireless work ethic. I
am highly adept at problem solving, but would like to gain greater experience of
working as a part of an effective team of other like-minded individuals. I
intend to pursue a career in this area: this placement will provide me with a
great opportunity to develop the skills needed for my future career. However,
the biggest factor in my favour is my absolute obsession with all aspects of
astrophysics, along with my ability to work effectively with other highly
motivated scientists.


\section*{Student 2}

\subsection*{Summary}
\begin{itemize}
    \item Passion for mathematics and software development
    \item Experience volunteering their time
\end{itemize}

\subsection*{Personal statement}

It has long been a firmly held aspiration of mine to become a software engineer.
With this in mind, I have elected to study Maths, Further Maths, Physics and
Politics for further education. My aim, to become a software developer, would
allow me to utilise my skills in Science and Maths and so realise a career that
meets my interest in computer sciences.\\

Maths has always intrigued me; its puzzle like nature presents a challenge to
embrace and decipher. Taking part in UKMT challenges has served to further
increase my love for maths and afforded me the opportunity to tackle
increasingly more demanding problems which, I understand, is a crucial part of
being a software engineer. I have a reasonable knowledge of Python from learning
it in class and online on websites such as Codecademy. Maths has always
intrigued me; its puzzle like nature presents a challenge to embrace and
decipher. It is often seen as a subject of black and white, a question of an
ultimate solution - a single, immutable conclusion. Indeed, it has these
aspects, but the journey to solving a maths’ problem is about more than its
solution. It is this discursive element and the possibilities that a maths’
questions sparks that has fostered my appreciation for maths. Taking part in
UKMT challenges has served to further increase my love for maths and afforded me
the opportunity to tackle increasingly more demanding problems. My part-time job
at Kumon Maths has allowed me to see the subject from a different perspective
and given me the challenge of explaining topics to younger students using
creative methods which allow the topic to be accessible to learners. Embracing
the challenges of leadership is one of my strongest skills.\\

I am currently a Master Cadet with a rank of Corporal in the Air
Training Corps, working towards the rank promotion of Sergeant. The ATC has
improved my leadership skills significantly with training in becoming a junior
non-commissioned officer. I was also a Senior Patrol Leader in Scouts, and
involvement in this group prompted my interest in computer science as a career
option, due to the networking skills garnered from numerous computer courses. I
have been the representative of my form for four consecutive years at in-house
council meetings, which necessitates my involvement in school decision making
processes, such as planning the building of a new coffee shop and designing the
extension of the changing rooms for boys and girls.\\

Attending a research experience with Nuffield Research Placements would benefit
me massively as it would allow me to gain skills specific to the maths and
computer science industry as well as the employability skills required for
real-life work. It would also increase my knowledge of the computer science
industry, allowing me to make better informed decisions about future career
choices. I think that I can make a positive contribution to the project because
I have participated in similar activities to this within a work experience
placement in Arup Engineering. This placement would benefit me massively as it
would allow me to mature as a student because of the responsibility and the
initiative I would have to take.\\

I feel that I can work as part of a team very well because of numerous
challenges that require excellent teamwork skills, for example parts of the
WBACC qualification and Duke of Edinburgh require a high ability to work as part
of a team and the ability to take instructions and carry them out. I am very
comfortable working individually, I can plan and organise tasks that are needed
to be done and I can adapt accordingly with the environment in which I work in.
I can act proactively and with integrity. My leadership qualities allow me to
take responsibility and initiative for my actions, this is something that I can
do very well.\\

Balancing my studies with extra-curricular activities has proved to be both
challenging and enjoyable and taught me to use my time efficiently. I play
volleyball and basketball regularly, and also play the violin to grade 5 level.
I have completed my Bronze Duke of Edinburgh; an inspiring experience which
improved both my independent decision making and teamwork skills. Throughout the
two days of the Duke of Edinburgh expedition we covered a distance of 30km. The
team faced many challenges that tested our perseverance and co-operation and
secured completion of the expedition. I have also volunteered at my local
library for six months, completing tasks such as locating resources for
visitors, stock-takes and taking note of new book arrivals. This opportunity
improved both my skills in self-confidence and organisation. I was also given
the chance to plan and deliver the Easter activities for young children which
proved to be an enjoyable challenge.\\

I am driven by a yearning to understand. This, together with my determination,
gives me assurance that I will benefit greatly from this work experience
placement.


\section*{Student 3}

\subsection*{Summary}
\begin{itemize}
    \item Has taken part in research placements with NASA and personal projects
    \item Focus is on medical/chemical applications
\end{itemize}

\subsection*{Personal statement}

I carry a highly enthusiast attitude towards Science and maths, showing a large
interest within the sector from a young age. I thrive in a learning community,
regularly wanting to learn. Growing up, I was surrounded and constantly inspired
by the research and information around me - exceedingly contributing to my
enthusiasm for the area. From simply researching the effects of a drug on the
human body to using technology to aiding training for medical students using
technology. From the simplest of issue from learning a simple topic in class,
then going onto researching it in depth and learning there is always a deeper
reason to a issue within science, is where my level of motivation towards the
subject comes from. My enthusiasm towards science is highlighted by me achieving
full UMS in GSCE science, and also achieving pupil of the Key Stage in Science
in Year 11.\\

Due to the fact that science is a subject that is constantly expanding, it
highly interests me and makes me very curious. I have a high interest in the
current news in biology and chemistry, growing from reading articles surrounding
the issue regarding medical research ranging to how new compounds are being
formed with properties which could be used in areas therefore creating endless
possibilities. This aspect of science highly interests me as new things can be
discovered everyday, and new species an form any second, hence the array of
possibilities are very interesting.\\

Through already participating in a small amount of research within the NASA
research project, I gained a large amount of knowledge from this hence driving
me to participate in other research projects as they allow you to physically
research instead of reading books. I would tremendously benefit from a research
placement as although I am proficiently confident in the carrier paths I would
like to take, the research placement would also give me a large insight into the
lab section of high level research, which could have a impact on my decision.
From partaking in simple experiments within the college lab, I had a large range
of questions for why the process occur the way they do, which is one key factor
in wanting to partake within the research placement as I will be very interested
in the topic I am researching.\\

Through Volunteering in a Charity Shop, it taught me how to work as a team, and
also gave me a large insight into what it was like helping people of all ages,
from little children to elderly people needing to purchase a item. Furthermore,
my ability to work as part of team was also developed while playing rugby for 3+
years, going to Germany and Belgium on tour as a team so high levels of teamwork
had to been shown and was developed.\\

I am also interested in Computing, I know a wide range of programming languages
from Python to using Javascript in Node-js. One aspect I am researching now is
the use of wide range programming languages to solve complicated algorithms
within Chemistry and Maths. This ranges from solving probability using the
elementary statical table using javascript math functions to researching
building interactive virtual environments for simulated training in medicine
using VRML and javascript. This is a area where computing immensely interests
me, hence why it is my fourth choice to study on my placement, where I can
integrate computing with solving math problems all the way to using it to
provide a safer environment to train medical students.. As for benefits, this
will eminently benefit me within any of the 4 subjects in the research
placement. It teaches you to problem solve by breaking a difficult problem into
many more solvable problems which vastly aid you with devoting a solution for
the problem. This shows a large level of commitment, persistence and initiative
into not give up when you face a issue which is difficult to solve.\\

Researching either of the 4 options, will help me to deepen my knowledge of the
human body, or finding out different variables with may effect a organisms or a
reaction, which will especially providing me with the skills necessary to work
in a medical research environment or within a lab situation, also hardening my
skills for reports, but most importantly hard-work and decimated research into a
difficult topic where I could learn a wide variety of issues. It is a wonderful
opportunity for me and it will aid me in discovering my interests and setting me
up for life long learning.

\end{document}
